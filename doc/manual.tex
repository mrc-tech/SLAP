\documentclass[10pt]{article}

\usepackage[a4paper,top=2.5cm,bottom=3.5cm,left=2.5cm,right=2.5cm]{geometry}
%\usepackage[T1]{fontenc}		%implementa nei font gli accenti (crea un casino facendo tutto pixellato)
\usepackage[utf8]{inputenc}		%importa i caratteri utf
\usepackage{inputenc}
\usepackage[italian]{babel}
\usepackage{amsmath,amssymb}	%per visualizzare la matematica (per text{})
%\usepackage{graphicx}			%per le figure
\usepackage{xcolor}				%per scrivere colorato XCOLOR per lstlisting colorato (se non c'è tikz)
\usepackage[toc,page]{appendix}	%per l'implementazione delle appendici
%\usepackage{subfig}				%per figure multiple in una stessa figure
%\usepackage{pdfpages}			%per includere pezzi di pdf
%\usepackage{float}				%per mettere le figure nel posto giusto
\usepackage{tikz}				%per le immagini vettoriali da LateX
\usepackage{listings} 			%Per inserire codice
\usepackage{pxfonts}			%per il bold nei lstlisting
%\usepackage{scrextend}			%per labeling
\usepackage{enumitem}			% per [noitemsep]


%\definecolor{mymauve}{rgb}{0.58,0,0.82}

\lstset { %
	language=C++,
	backgroundcolor=\color{black!5}, % set backgroundcolor
	basicstyle=\footnotesize\ttfamily, % basic font setting
	frame=shadowbox,
	breaklines= true, % va a capo automatico
	numbers=left, % numera le righe
	commentstyle=\color{gray}, % commenti grigi
	stringstyle=\color{mymauve}, % string literal style
	numberstyle=\footnotesize\ttfamily, %formato dei numeri a lato delle righe
	keywordstyle=\color{black}\bfseries, % style for keywords
}




\begin{document}


\title{\textbf{SLAP}\\ Simple Linear Algebra Package}
\author{Andrea Marchi}

\maketitle

\tableofcontents



\section{Data type}

Le matrici sono salvate in array monodimensionali \textit{row-major}, ovvero che i dati sulle righe sono sequenziali. Quindi nella indicizzazione degli elementi della matrice con $n$ righe e $m$ colonne ($\mathbb{R}^{n\times m}$) si usa la formula:
\begin{lstlisting}
matrix[i][j] = array[i*m + j]
\end{lstlisting}
Le righe vanno da $0$ a $n-1$, mentre le colonne vanno da $0$ a $m-1$.

Alcune librerie usano la notazione \textit{column-major}, ovvero con indicizzazione \verb|array[j*n+i]|. Avere una funzione che organizza i dati della matrice in colonne o in righe è comodo per quanto riguarda l'ottimizzazione della cache di lettura dei dati sequenziali dalla RAM alla CPU.

Il tipo di dati (essendo in C) è una struttura (\verb|struct|) e la definizione cambia al variare del tipo di dati base (il C non permette l'uso di template).
La struttura base è:
\begin{lstlisting}
typedef struct _matd{
	unsigned int n_rows;
	unsigned int n_cols;
	double *data; // row-major matrix data array
} matd;
\end{lstlisting}
dove la lettera finale indica il tipo di variabile usata, in questo caso \verb|double|. I tipi di dati che ha senso utilizzare nella libreria sono:
\begin{description}[noitemsep]
\item[\texttt{d}] virgola mobile a doppia precisione (\verb|double|)
\item[\texttt{f}] virgola mobile (\verb|float|)
\item[\texttt{i}] intero (\verb|int|)
\item[\texttt{b}] byte (\verb|unsigned char|)
\end{description}

Per allocare la memoria e liberarla (sempre liberare la memoria dopo averla allocata):
\begin{lstlisting}
matd* new_matd(unsigned int num_rows, unsigned int num_cols)
{
	int i;
	// create a new double matrix
	if(num_rows == 0) { /*SLAP_ERROR(INVALID_ROWS);*/ return 0; } // dovrebbe ritornare NULL
	if(num_cols == 0) { /*SLAP_ERROR(INVALID_COLS);*/ return 0; } // dovrebbe ritornare NULL
	
	matd *m = calloc(1, sizeof(*m)); // allocate space for the struct
	// CONTROLLARE LA MATRICE CREATA ( SLAP_CHECK(m) )
	m->n_rows = num_rows;
	m->n_cols = num_cols;
	m->data = calloc(m->n_rows*m->n_cols, sizeof(*m->data));
	// CONTROLLARE I DATI CREATI ( SLAP_CHECK(m->data) )
	for(i=0; i<num_rows*num_cols; i++) m->data[i] = 0; // set to zero
	
	return m;
}

void free_mat(matd *matrix)
{
	free(matrix->data); // delete the data
	free(matrix); // delete the data structure
}
\end{lstlisting}

Come setters e getters non potendo usare le operation del C++ e non riuscendo a fare qualcosa di funzionante e decente con le macro\footnote{Usare le macro mi permetterebbe di risparmiare tempo nella allocazione dei parametri delle funzioni. Negli algoritmi potrei usare l'accesso diretto all'array \texttt{data}.} uso le funzioni
\begin{lstlisting}
double matd_get(matd matrix, unsigned int row, unsigned int col) {
	return matrix.data[row*matrix.n_cols + col]; // row-major
}
void matd_set(matd matrix, unsigned int row, unsigned int col, double val) {
	matrix.data[row*matrix.n_cols + col] = val; // row-major
}
\end{lstlisting}
\textcolor{red}{Dovrei controllare che l'accesso sia corretto (che non cerchi di scrivere/leggere dati non allocati).}



\section{Basic operations}

\textcolor{red}{Mettere un pannello (anche una tabella) che riassume le operazioni, il nome delle funzioni e come si usano.}

\subsection{Equality}

\subsection{Addition and sottration}

\subsection{Multiplication}

\subsubsection{Scalar-matrix multiplication}
\subsubsection{Vector-matrix multiplication}
\subsubsection{Matrix-matrix multiplication}







\newpage

\appendix

\section{Random numbers}

\textcolor{red}{Algoritmi di generazione dei numeri casuali}

\subsection{Sampling from Gaussian distribution}

\textcolor{red}{metodi di sampling descritti sul mio quaderno dei ponti integrali, con qualche indicazione in piu`}



\end{document}